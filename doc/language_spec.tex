\documentclass[a4paper,11pt,pdftex]{article}

\usepackage{a4wide}
\usepackage{xspace}
\usepackage{url}

\newcommand{\implname}{Spannerlib\xspace}
\newcommand{\ie}{IE\xspace}


\begin{document}

\title{\implname\\Design Document}
\date{\today}

\maketitle

The \emph{\implname} is intended to become of an implementation that allows
declarative expression of Information Extraction tasks. \implname is based on
the (rule-based) approach to Information Extraction as expressed by
XLog~\cite{SDNR07}, SystemT~\cite{CKLRRV10}, and DeepDive~\cite{DBLP:journals/cacm/ZhangRCSWW17}. In this approach, complex extractor logic is expressed by combining \emph{primitive extractors} (e.g., tokenizers, dictionaries, part-of-speech taggers, regular expressions matchers, \dots) through algebraic combinators.

In particular, \implname implements the theoretical framework of Document
Spanners~\cite{FaginKRV15}, which formalizes the rule-based approach to Information
Extraction cited above. In this framework, a \emph{document spanner} (or just
spanner for short) is any function that takes a document as input and yields a
relation over the \emph{spans} (text intervals) of this input.  Various
languages for expressing document spanners have recently been studied. \implname
will implement \textsf{spannerlog}~\cite{DBLP:conf/icdt/PeterfreundCFK19}, which is the closure of  regular expression formulas under recursive Datalog programs.

This design document is intended to describe the concrete syntax of \ie programs in version 0.1 of the \implname. 

\section{Syntax and Semantics}
\label{sec:syntax}

\paragraph*{Primitive extractors.} The only primitive extractors that we intend to support in version 0.1 are \emph{functional regex formulas}. Essentially, a regex formula is a regular expression with capture variables. In the literature~\cite{FaginKRV15} such functional regex formulas are often given a very simple syntax, e.g.,
\[ \gamma ::= \emptyset \mid \varepsilon \mid \sigma \mid \gamma \vee \gamma \mid \gamma \cdot \gamma \mid \gamma^* \mid x\{\gamma\},\]
where $\varepsilon$ represents the empty string, $\sigma$ ranges over alphabet letters, and $x$ ranges over variable names.
While this minimalistic syntax is appealing for theoretical investigation, it is
cumbersome to use in practice. Indeed, regular expressions in practice have many
convenient abbreviations (see e.g.,
\url{https://www.pcre.org/current/doc/html/pcre2pattern.html}).


As a concrete syntax for regex formulas, it is proposed to use the syntax of
Perl regular expressions, where we interpret \emph{named capture groups} to
indicate variable capture. (See
\url{https://www.regular-expressions.info/named.html} for a discussion on named
capture groups.) Unnamed capture groups are not supported ignored. Backreferences are not allowed (since this allows recognition of non-regular languages). As usual, functional regex formulas are given a \emph{all-match} semantics instead of the unique match semantics supported by Python/Perl/POSIX regular expressions. We enforce functionality by syntactic checks.

While in principle Perl regular expression allow naming capture groups using both the .NET syntax and the python syntax, we choose the .NET syntax.


Some examples:
\begin{itemize}
\item The expression \verb!a(?<x>b)c(?<y>d)! has two named capture groups: \verb!(?<x>b)! and \verb!(?<y>d)! where the former matches $b$ (binding the matched span to $x$, and the latter matches $d$ (binding the matched span to $y$). As such, matching the entire expression against the document $abcd$ is successful, and outputs the singleton relation with the tuple $t$ where $t(x) = [1,2>$ and $t(y) = [3,4>$.
\item  The expression \verb![.*(?<year>\d\d\d\d).*(?<amount>\d+)\sEUR! matches the input document \verb!In 2019 we earned 2000 EUR!, extracting 2019 in \verb!year!, and 2000 in \verb!amount!
\item The above examples use the .NET syntax. The Python syntax would be  \verb!a(?P<x>b)c(?P<y>d)! and \verb![.*(?P<year>\d\d\d\d).*(?P<amount>\d+)\sEUR!, respecitvely.
\end{itemize}

Other important decisions:
\begin{itemize}
\item We implicitly assume that a regular expression formula is prefixed by $.*$ and suffixed by $.*$ (which means it will find any  occurrence of the regex in the string), \emph{unless} it starts with the anchors (\verb!^! and  \verb!$!) that ask it to match the entire string? This means that in the second example above, there was no need  explicitly add \verb!.*! at the beginning to make sure that matching could start anywhere.
\item We assume that the input string is UTF8 encoded. This is compatible with ASCII, but also allows to process international text.
\end{itemize}


\paragraph*{Rules, Programs and Queries}
\label{sec:rules}

In version 0.1, a \emph{program} will essentially be a (standard) Datalog program where:
\begin{itemize}
\item all predicate symbols are \emph{intensional}
\item regex formulas function as \emph{extensional} relations
\item all variables are of \emph{span} type. (I.e., every variable can only store spans, not strings or other data).
\end{itemize}
This corresponds to what is described in ``Recursive Programs for Document Spanners'', published in ICDT 2019.

\smallskip
Syntactic considerations:
\begin{itemize}
\item It is proposed to use \verb!<-! for separating rule heads from rule bodies.
\item It is proposed to not impose any syntactic restrictions on relation and variable names.  (In some Datalog variants, variable names must start with a capital \dots).
\item Inside a rule body, a regex formulas is denoted as follows:  $\textsf{RGX}\texttt{<}f\texttt{>}(x_1,\dots,x_n)$. Here, $f$ is a regex formula and $x_1, \dots, x_n$ is a listing of all the variables in $f$. (Note that these variables may appear in a different order in $f$).
\item To specify that a regex formula inside a rule needs to executed on a specific input document,  we  use  epxressions like \[ \verb!extract RGX<f>(x_1,\dots,x_n)  from  d!\]  that specifies the regex formula as well as where to extract from. Here, $d$ is allowed to be (1) a constant string or (2) a string variable. We allow defining and instantiating string variables by the syntax
  \[ d= \textsf{'some string'},\]
  (which initializes $d$ to the corresponding string literal, or \[ d = \textsf{read('data/a.txt')}\]
  which loads the contents of the file \verb!data/a.txt! into string variable $d$.

% \item To specify that a regex program needs to be run on a given input document we allow special expressions like \[ \verb!extract R(x,z,_) from d!\]  that specifies the target relation to extract, as well as where to extract from. Here, $d$ is allowed to be (1) a constant string or (2) a string variable. We allow defining and instantiating string variables by the syntax
%   \[ d= \textsf{'some string'},\]
%   (which initializes $d$ to the corresponding string literal, or \[ d = \textsf{read('data/a.txt')}\]
%   which loads the contents of the file \verb!data/a.txt! into string variable $d$.
\end{itemize}

\smallskip


Left for future work:
\begin{itemize}
\item Provisions for modularity (e.g., grouping rules in modules, having namespaces to avoid nameclashes etc) will not be included in version 0.1, and are postponed to a later version.
\item Syntactic sugar for things that are expressible, but cumbersome to expres (e.g., the $\varsigma$ operator, same length, prefix, \dots)  will not be included in version 0.1, and are postponed to a later version.
\end{itemize}

 


\bibliographystyle{abbrv}
\bibliography{biblio}











\end{document}

%%% Local Variables:
%%% mode: latex
%%% TeX-master: t
%%% End:
